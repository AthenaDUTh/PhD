% !TeX root = ../../thesis.tex
\chapter{Abstract}                                 \label{ch:abstract}

When comparing modern smartphone cameras to those from the 2000s, the improvements in image quality, including resolution, low-light sensitivity, and color accuracy, are remarkable. These improvements can be partially attributed to technological advancements in the design and fabrication of CMOS image sensors, the camera component responsible for converting incoming light into electrical signals. In this context, the emergence of color-splitting nano-photonic structures as an alternative to conventional color filter arrays marks a promising advancement that may accelerate the continuous scaling of pixel pitch in image sensors.

Nevertheless, the continuous reduction of pixel pitch has the adverse effect of increased optical and electrical crosstalk, exacerbated by the low absorption coefficient of silicon, which is used as the photoconverting material. For this reason, thin-film photodetectors emerge as a promising alternative. Having an absorption coefficient that is nearly one order of magnitude higher than that of silicon in the visible spectrum, thin-film photodetectors enable comparable photon absorption within just a few hundred nanometers of thickness. Taking into account the requirements for high response speed and compatibility with high-temperature semiconductor fabrication processes, inorganic metal halide perovskites are selected as a suitable candidate. Additional requirements for scalable fabrication motivate the use of the vacuum thermal evaporation as the perovskite deposition approach. 

Even though solution-processed, hybrid organic-inorganic perovskites have been extensively investigated over the past 15 years in the context of solar cell applications, the development of all-inorganic and solely vacuum deposited perovskite photodiodes remains elusive. This observation was the main motivation of the present work, which has been approached from three different perspectives.

First, a comprehensive characterization of thermally co-evaporated \ch{CsPbI_2Br} thin films is conducted. Special emphasis is given on investigating the impact of the post-deposition thermal annealing step, whose real-time effect is studied by means of temperature-dependent spectroscopic ellipsometry. The latter is demonstrated to be an accessible, reliable, and low-cost approach to gain insights not only on the film's optical constants, but also its temperature-dependent expansion, roughness increase, and phase evolution. 

Second, the \ch{CsPbI_2Br} layer is integrated into a vacuum-deposited and all-inorganic photodiode structure. The choice of the electron transport layer (ETL) is revealed to be critical for the diodes' reverse bias stability, performance variability, and response speed. Eventually, the adoption of an ETL that consists of a fullerene and a metal oxide bi-layer, along with the careful tuning of the respective thicknesses, leads to a threefold optimization, through a minimization of interface defects, an elimination of shunt pathways, and a reduction of the diode's RC constant. 

Third, a high-throughput experimentation approach is introduced to investigate the ambient phase stability of \ch{CsPbI_2Br} thin films and overcome challenges related to extreme batch-to-batch or even within-batch variations. We show that by depositing a broad compositional gradient of the perovskite across a 6'' wafer, the optimal range of precursor molar ratios can be identified, significantly enhancing the films' ambient stability from a few hours to several months.

%%%%%%%%%%%%%%%%%%%%%%%%%%%%%%%%%%%%%%%%%%%%%%%%%%
% Keep the following \cleardoublepage at the end of this file, 
% otherwise \includeonly includes empty pages.
\cleardoublepage

% vim: tw=70 nocindent expandtab foldmethod=marker foldmarker={{{}{,}{}}}
