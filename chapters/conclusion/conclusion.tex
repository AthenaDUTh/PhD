% !TeX root = ../../thesis.tex
\chapter{Conclusions and Outlook}\label{ch:conclusion}

The present thesis is a contribution to the development and optimization of perovskite-based photodiodes aimed for ultra-high imaging applications. The inception of this study was driven by the development of imec's nano-photonic color splitting technology that opens the way for the development of CMOS image sensors with pixel pitch below 0.5 $\mu m$. The requirement for compatibility with high-temperature fabrication processes directed the use of all-inorganic metal halide perovskites, and specifically the \ch{CsPbI_2Br} compositional variation, owing to the optimized trade-off between optical bandgap and phase stability under ambient conditions. Additionally, the need for scalable and defect-free fabrication motivated the use of vacuum thermal co-evaporation for the deposition of \ch{CsPbI_2Br} films. Besides the perovskite layer itself, the complete photodiode stack was fabricated using inorganic and vacuum-deposited materials, setting our work apart from the vast majority of reports in the same field. 

The simultaneous elimination of solution processing and organic compounds from the development of the photodiode stack gave rise to an intricate set of challenges. For instance, solution processing, and specifically spin coating, is widely adapted as a fabrication method, owing to low costs, great flexibility in terms of compatible materials, and rapid prototyping. As a result, relying solely on vacuum processing techniques minimizes, in a sense, these advantages, imposing a relatively limited gamut of compatible materials and slower processing time-frames. Additionally, vacuum-processing relies on expensive and complex equipment, that causes long down-times when a component needs to be fixed or replaced. Nevertheless, the adoption of vacuum-processing for perovskite layers eliminates user-to-user variabilities, that are extremely common during manual spin-coating processing, and ensures high quality, large scale depositions. On the other hand, the remarkable progress that has been achieved over the past 15 years in the field of perovskite-based solar cells, light emitting diodes or photodetectors could be partially attributed to the multifaceted use of organic compounds, which ranges from their adoption as components of the perovskite lattice and transport layers to their addition as passivation, interface modification, or encapsulation layers. However, despite the seeming disadvantages"of excluding the use of organic compounds, the benefits in developing high-temperature tolerant devices are critical for their integration in fab-compatible process flows and their large scale implementation. 

Since each chapter is already concluded with specific insights and metrics from the relevant endeavors, the goal of the following paragraphs is to take a step back and provide a bigger-picture overview.

Our research endeavors started in Chapter~\ref{ch:material_properties}, where the standardization of the experimental methodology and characterization process was presented. The "soft", ionic nature of metal halide perovskites necessitates the cautious interpretation of experimental results, both on material and on device level. In order to ensure consistency of results across different devices and fabrication runs it was deemed necessary to take into account the effects of biasing or illumination stress. As a result, performing steady-state measurements, such as that of current density or EQE as a function of time, on "fresh" devices might be the most reliable way to compare different samples and extract meaningful conclusions. 


Followingly, Chapter~\ref{ch:ellipsometry} continued with a comprehensive evaluation of the material properties of co-evaporated \ch{CsPbI_2Br} thin films and the impact of the post-deposition annealing step. Interestingly, the as-deposited \ch{CsPbI_2Br} films where already in the black perovskite phase, indicating their suitability for alternative applications where a post-deposition annealing step is not possible. Besides the standard techniques that are typically employed in this field, we quickly identified the need for an easily accessible, low-cost, reliable, and non-destructive characterization technique that could provide insights into the real-time annealing effect on perovskite thin films. In this context, temperature-dependent spectroscopic ellipsometry (SE) emerged as a promising candidate, which fulfilled all of the above-mentioned requirements. Unlike previous studies that employed incremental or decremental temperature steps during temperature-dependent SE measurements, we identified the advantages of using a continuous heating ramp, which prevents the masking of rapid material transformations. This was further facilitated by the development of a novel dynamic model that can describe the material's structural, morphological, and optical evolution in real time without compromising neither the accuracy nor the speed of the simulation. The results of the simulation process provided new insights on the roughness increase and grain coalescence of the evaporated perovskite layer as a function of increasing temperature, while revealing certain characteristic signatures that serve as reliable indicators of phase transitions.


Once the properties of the material where fully characterized, the \ch{CsPbI_2Br} perovskite was integrated into bottom- and top-illuminated photodiodes in Chapter~\ref{ch:transport_layer}. Starting with the bottom-illuminated PePD structure that allowed faster prototyping, the defect-tolerant nature of the perovskite lattice was emphasized. Efforts in tuning the perovskite annealing conditions or precursor molar ratio had minimal impact on the PePD performance, leading to variations that fall within the expected range of batch-to-batch fluctuations. On the contrary, the role of the ETL proved to be way more impactful on the reliability and speed of the perovskite photodiodes. Specifically, the deposition of \ch{TiO2} via e-beam evaporation was associated with the creation of interface defect states that promote reverse bias breakdown via trap-assisted tunneling of holes into the perovskite layer. The specific mechanism could be easily overlooked if relying solely on fast dynamic measurements for the estimation of the PePD's dark current, highlighting once again the importance of steady-state measurements. On the other hand, when evaluating the use of \ch{C_{60}}, the PePD was associated with increased variability, even among devices fabricated on the same sample, despite the widespread use of \ch{C_{60}} as an electron selective layer. This finding was attributed to the penetration of the metallic contact through the fullerene and the formation of metallic shunts that can be reduced, albeit not eliminated, with increasing the ETL thickness. The implementation of a \ch{C_{60}}-\ch{TiO_2} bilayer was sufficient to mitigate both issues, while increasing  the thickness of the \ch{C_{60}} layer was found to have a positive impact on the PePD's speed, through an extension of the diode's depletion width and a consequent reduction of its RC constant. Lastly, the optimized top-illuminated PePD was exposed at 250 \degree C for 60 minutes to evaluate the high-temperature tolerance of the stack, with the results validating our choice of developing an all-inorganic diode configuration.

Lastly, Chapter~\ref{ch:stability} expanded on the ambient stability of thermally evaporated \ch{CsPbI_2Br} thin films. Despite the repeatable optoelectronic performance of PePDs, even with large variations in the composition of the perovskite layer, the thin film ambient stability demonstrated drastic fluctuations, even among, theoretically, similar samples. These fluctuations, which were initially in the range of a few hours to several days, highlighted a known limitation of evaporated perovskites, where the deposited composition can substantially differ from the nominal one. In order to overcome the lack of compositional repeatability, we employed a high-throughout co-evaporation approach, by de-activating the substrate rotation during deposition. This way a large gradient of precursor ratios was deposited on a 6'' wafer, revealing that the CsBr to \ch{PbI_2} molar ratio should lie between 1.15:1.00 and 1.35:1.00 for optimal stability, that extends over several months. In practice, considering a constant \ch{PbI_2} deposition rate of 0.4~\AA/s, the CsBr rate should be constrained between 0.3 and 0.39~\AA/s for the perovskite composition to be within the desired limits. This means that inability to control the deposition rate with such precision would have adverse effects on the consistency of the films' ambient stability. 

Although significant progress has been achieved, many challenges still need to be surmounted before perovskite-based photodiodes become commercially competitive to conventional silicon image sensors. To begin with, the integration of the optimized top-illuminated PePDs on top of imec's Si ROIC platform is necessary to extract critical figures of merit, related to signal conversion (conversion gain, dynamic range, full-well capacity, etc.) and spatial uniformity. This will allow a more robust comparison with Si image sensors. In any case, further efforts to reduce the device's dark current are essential to improve its specific detectivity. Such efforts could involve technology-oriented approaches, such as the introduction of pixel edge cover layers, or material-oriented approaches, such as the optimization of the diode's energetic landscape or the introduction of passivation layers. 


Beyond optimizing the PePD's optoelectronic performance, improvements in the perovskite's ambient stability and toxicity are also essential. Even though the advancements presented in the scope of this thesis are already a major leap forward for prototyping-oriented endeavors, the transition to commercial applications requires a more thorough reliability study of the films' ambient stability. In this context, additional efforts could be directed towards the development of compatible thin-film encapsulation layers. Lastly, an aspect that has not been discussed in this work, yet is critical for the commercialization of perovskite-based devices of any kind, is the presence of lead and the associated health risks. Alternative perovskite compositions that replace lead with tin (Sn), bismuth (Bi) or antimony (Sb) have already appeared, however they still lack in maturity compared to Pb-based perovskites. Despite the considerable challenges, coordinated efforts by research groups worldwide could initially target niche applications of perovskite-based photodetectors, eventually paving the way for broader commercial adoption.


%%%%%%%%%%%%%%%%%%%%%%%%%%%%%%%%%%%%%%%%%%%%%%%%%%
% Keep the following \cleardoublepage at the end of this file, 
% otherwise \includeonly includes empty pages.
\cleardoublepage

% vim: tw=70 nocindent expandtab foldmethod=marker foldmarker={{{}{,}{}}}
