% !TeX root = ../../thesis.tex
\chapter{Beknopte samenvatting}

Wanneer we moderne smartphonecamera's vergelijken met die uit de jaren 2000, zijn de verbeteringen in beeldkwaliteit opmerkelijk. De resolutie, gevoeligheid bij weinig licht en kleurnauwkeurigheid zijn met de jaren al sterk geoptimaliseerd. Deze verbeteringen kunnen deels worden toegeschreven aan technologische vooruitgang in het ontwerp en de fabricage van CMOS-beeldsensoren, het onderdeel van de camera dat verantwoordelijk is voor het omzetten van binnenkomend licht in elektrische signalen. In deze context markeert de opkomst van kleursplitsende nanofotonische structuren als alternatief voor conventionele kleurfilter arrays een veelbelovende ontwikkeling die de continue schaling van de pixelgrootte in beeldsensoren zou kunnen versnellen.

Toch heeft de continue reductie van de pixelafstand het nadelige effect van verhoogde optische en elektrische overspraak, verergerd door de lage absorptiecoëfficiënt van silicium, dat als fotonisch conversiemateriaal wordt gebruikt. Om deze reden komen dunne film fotodetectoren naar voren als een veelbelovend alternatief. Met een absorptiecoëfficiënt die bijna een orde van grootte hoger is dan die van silicium in het zichtbare spectrum, maken dunne film fotodetectoren vergelijkbare fotonabsorptie mogelijk binnen een dikte van slechts enkele honderden nanometers. Rekening houdend met de eisen voor een hoge reactiesnelheid en compatibiliteit met hoge temperatuur halfgeleider fabricageprocessen, worden anorganische metaalhalideperovskieten geselecteerd als een geschikte kandidaat. Aanvullende eisen voor schaalbare fabricage motiveren het gebruik van vacuüm thermische verdamping als perovskiet depositiemethode. Alhoewel deze gedeponeerd worden via een oplossing, hybride organisch-anorganische perovskieten werden de afgelopen 15 jaar uitgebreid onderzocht in de context van zonneceltoepassingen. Toch blijft de ontwikkeling van volledig anorganische en uitsluitend vacuümgedeponeerde perovskietfotodiodes onbereikbaar. Deze observatie vormde de belangrijkste motivatie voor het huidige werk, dat vanuit drie verschillende perspectieven is benaderd.

Ten eerste wordt een uitgebreide karakterisering uitgevoerd van thermisch co-verdampte dunne CsPbI2Br-films. Speciale aandacht wordt besteed aan het onderzoeken van de impact van de thermische annealing na de depositie, waarvan het real-time effect wordt bestudeerd met behulp van temperatuur afhankelijke spectroscopische ellipsometrie. Deze laatste methode blijkt een toegankelijke, betrouwbare en goedkope manier te zijn om niet alleen inzicht te krijgen in de optische constanten van de film, maar ook in de temperatuur afhankelijke uitzetting, ruwheid toename en fase-evolutie.

Ten tweede is de CsPbI2Br-laag geïntegreerd in een vacuüm gedeponeerde en volledig anorganische fotodiode structuur. De keuze van de elektronen transportlaag (ETL) blijkt cruciaal te zijn voor de stabiliteit van de sperbias, de prestatie variabiliteit en de responssnelheid van de diodes. Uiteindelijk leidt de keuze voor een ETL, bestaande uit een fullereen- en een metaaloxide-dubbellaag, samen met de zorgvuldige afstemming van de respectievelijke diktes, tot een drievoudige optimalisatie: minimalisatie van interfacedefecten, eliminatie van shuntpaden en een verlaging van de RC-constante van de
diode.

Ten derde introduceren we een high-throughput experimentenbenadering om de stabiliteit van de omgevingsfase van dunne CsPbI2Br-films te onderzoeken en de uitdagingen te overwinnen die gepaard gaan met extreme variaties van batch tot batch of zelfs binnen batches. We tonen aan dat door een brede samenstellingsgradiënt van perovskiet over een 6" wafer te deponeren, het optimale bereik van de molaire verhoudingen van de precursor kan worden geïdentificeerd, waardoor de stabiliteit van de films in de omgevingsfase aanzienlijk wordt verbeterd van enkele uren tot enkele maanden.


%%%%%%%%%%%%%%%%%%%%%%%%%%%%%%%%%%%%%%%%%%%%%%%%%%
% Keep the following \cleardoublepage at the end of this file, 
% otherwise \includeonly includes empty pages.
\cleardoublepage

% vim: tw=70 nocindent expandtab foldmethod=marker foldmarker={{{}{,}{}}}
