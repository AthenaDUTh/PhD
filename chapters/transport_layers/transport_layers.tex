% !TeX root = ../../thesis.tex
\chapter{Electron Transport Layer Optimization}\label{ch:transport_layer}


One of the most critical factors that indicates the performance of photodetectors is the dark current density (Jd), which constitutes the primary limitation on the device’s detectivity. Previous reports have shown that careful optimization of the ETL and HTL is critical for minimizing Jd in perovskite-based photodetectors. This was achieved by eliminating shunt pathways25, increasing the carrier injection barrier26,27, and preventing interfacial thermal charge generation28,29. However, most reports typically restrict the operational voltage range to approximately -0.5 V, despite the potential for higher carrier extraction speeds at greater reverse biases. This is because extensive reverse biasing of perovskite heterojunctions is known to have adverse effects on the device performance, including the risk of device breakdown. This manifests as a multiple-orders-of-magnitude increase in Jd and has been attributed to the creation of defect states in the perovskite bulk that act as charge recombination centers30. Specifically, it was shown that under reverse bias ions accumulate at the perovskite/ETL interface, facilitating the tunneling of holes into the perovskite bulk. Recombination centers are thereafter generated by the oxidation of halides to natural halogens31. Diffusion of iodine into the fullerene-based ETL was also shown to be an aftermath of reverse biasing32. Various breakdown mitigation strategies were proposed, including the introduction of a hole-blocking layer at perovskite/ETL interface33,34, the use of electrochemically inert electrodes34,35, as well as the deposition of thick planarizing polymer HTLs35. It is important to mention that all aforementioned studies focus on perovskite solar cells rather than photodetectors, where reverse biasing falls outside the “normal” operation conditions and may arise when a cell under shadow has to curry the current generated by the neighboring, illuminated cells.


%%%%%%%%%%%%%%%%%%%%%%%%%%%%%%%%%%%%%%%%%%%%%%%%%%
% Keep the following \cleardoublepage at the end of this file, 
% otherwise \includeonly includes empty pages.
\cleardoublepage

