% !TeX root = ../../thesis.tex
\chapter{Popularized Abstract}\label{ch:popabstract}

The quality of commercial cameras has come a long way in the past decades, yet opportunities to make them smaller and better still exist. Such an opportunity arises in exploring alternative light-detecting materials. Traditionally, silicon has been used for this purpose; however, it requires a relatively large thickness to capture all information carried by incoming light. When pixels of small area but large thickness are packed closely together, some of the light information can leak between them, reducing image clarity. Our research explores the use of an alternative material that requires 1/10th of silicon's thickness to capture the same information, thereby eliminating light or signal leaks between adjacent pixels. This material belongs to the family of metal halide perovskites. 

To successfully replace a material as well-established as silicon, stringent criteria must be fulfilled. For instance, the light-detecting device should be able to withstand high semiconductor processing temperatures and to be produced in large scales. The former condition is met with the use of only inorganic materials, while the latter is fulfilled with the employment of vacuum-deposition approaches. As a result, the study and optimization of inorganic perovskite photodetectors, fabricated exclusively through vacuum-deposition techniques, has been the main objective of this thesis. 

The topic has been approached from three different perspectives. Firstly, we extensively evaluate the quality of the vacuum-deposited perovskite layer and investigate the impact of a thermal treatment step on its properties. Secondly, we incorporate the perovskite layer in photodetecting devices and explore pathways to improve their reliability, repeatability, and speed. Lastly, we adopt a method that allows us to test a large number of processing conditions at once, helping us identify the ones that ensure the perovskite layer remains stable in humid environments.


%%%%%%%%%%%%%%%%%%%%%%%%%%%%%%%%%%%%%%%%%%%%%%%%%%
% Keep the following \cleardoublepage at the end of this file, 
% otherwise \includeonly includes empty pages.
\cleardoublepage
