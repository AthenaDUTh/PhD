% !TeX root = ../../thesis.tex
\chapter{Gepopulariseerde Samenvatting}\label{ch:popabstractnl}

De kwaliteit van commerciële camera's heeft de afgelopen decennia een enorme vooruitgang geboekt, maar er bestaan nog steeds mogelijkheden om ze kleiner en beter te maken. Zo'n kans doet zich voor bij het verkennen van alternatieve licht detecterend materialen. Traditioneel wordt hiervoor silicium gebruikt, maar het vereist een relatief grote dikte om alle informatie van het binnenkomende licht vast te leggen. Wanneer pixels met een klein oppervlak maar een grote dikte dicht op elkaar worden geplaatst, kan er lichtinformatie tussen de pixels lekken, waardoor de beeldhelderheid afneemt. Ons onderzoek bestudeert het gebruik van een alternatief materiaal dat 1/10e van de dikte van silicium nodig heeft om dezelfde informatie vast te leggen, waardoor licht- of signaallekken tussen aangrenzende pixels worden geëlimineerd. Dit materiaal behoort tot de familie van metaalhalideperovskieten.

Om een zo gevestigd materiaal als silicium succesvol te vervangen, moeten strenge criteria worden nageleefd. Zo moet de lichtdetector bestand zijn tegen hoge temperaturen bij de verwerking van halfgeleiders en op grote schaal geproduceerd kunnen worden. De eerste voorwaarde wordt voldaan door uitsluitend anorganische materialen te gebruiken. De tweede voorwaarde wordt voldaan door middel van vacuümdepositie. De studie en optimalisatie van anorganische perovskietfotodetectoren, die uitsluitend met behulp van vacuümdepositietechnieken worden vervaardigd, was dan ook het hoofddoel van dit proefschrift.

Het onderwerp is vanuit drie verschillende perspectieven benaderd. Ten eerste evalueren we uitgebreid de kwaliteit van de vacuümgedeponeerde perovskietlaag en onderzoeken we de impact van een thermische behandelingsstap op de eigenschappen ervan. Ten tweede integreren we de perovskietlaag in fotodetectoren en onderzoeken we mogelijkheden om hun betrouwbaarheid, herhaalbaarheid en snelheid te verbeteren. Ten slotte passen we een methode toe waarmee we een groot aantal verwerkingsomstandigheden tegelijk kunnen testen, waardoor we de omstandigheden kunnen identificeren die ervoor zorgen dat de perovskietlaag stabiel blijft in vochtige omgevingen.


%%%%%%%%%%%%%%%%%%%%%%%%%%%%%%%%%%%%%%%%%%%%%%%%%%
% Keep the following \cleardoublepage at the end of this file, 
% otherwise \includeonly includes empty pages.
\cleardoublepage
