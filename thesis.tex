\documentclass[showinstructions,showlabels,coverfontpercent=100]{adsphd}
%\documentclass[croppedpdf,showinstructions]{adsphd}
%\documentclass[online]{adsphd}
%\documentclass[print]{adsphd}

\usepackage{nomencl}   % For nomenclature
\usepackage[style=long,number=none]{glossary} % For list of abbreviations

% !!!!!!!!!!!!!!!!!!!!!!!!!!!!!!!!!!!!!!!!!!!!!!!!!!!!!!!!!!!!!!!!!!!
% !!                                                               !!
% !!  WARNING: do not remove the following lines between           !!
% !!  "%%% COVER: Settings %%%" and "%%% COVER: End settings %%%"  !!
% !!                                                               !!
% !!!!!!!!!!!!!!!!!!!!!!!!!!!!!!!!!!!!!!!!!!!!!!!!!!!!!!!!!!!!!!!!!!!

%%% COVER: Settings %%%
\title{The Title of Your PhD Dissertation}
%\subtitle{Multiscale computing for Dummies}

\author{Your}{Name}

\supervisor{Prof.~dr.~ir.~B.~Leader}{}
\supervisor{Prof.~dr.~ir.~S.~Second-Leader}{}
\president{Prof.~dr.~ir.~The~Chairman}
\jury{Prof.~dr.~ir.~The~One\\
      Prof.~dr.~ir.~The~Other
}
\externaljurymember{Prof.~dr.~External~Jurymember}{Far Away}

% For the correct values that you have to fill in underneath, check:
% Faculty of Engineering Science:
%   see https://onderwijsaanbod.kuleuven.be/opleidingen/e/SC_50046650.htm
% Other faculties:
%   see https://onderwijsaanbod.kuleuven.be/opleidingen/e/index.htm
\phddegree{Engineering Science: Computer Science} % "Doctor of ..."
\faculty{Faculty of Engineering Science}
\department{Department of Computer Science}

% Your research group within the department
% e.g. iMinds-DistriNet, Scientific Computing Group, ...
\researchgroup{XXXXX}
\website{http://www.XXXXX.cs.kuleuven.be} % Leave empty to hide
\email{first.last@cs.kuleuven.be} % Leave empty to hide

\address{Celestijnenlaan 200A box 2402}
% \addresscity{B-3001 Leuven} % This is the default value. Note
                              % that 'B-3001 Heverlee' is _incorrect_!!
                              % /[https://www.kuleuven.be/communicatie/marketing/intranet/huisstijl/taalgebruik.html]

\date{January 2016}
\copyyear{2016}
%\udc{XXX.XX}            % UDC, deposit number and ISBN are no longer necessary.
%\depot{XXXX/XXXX/XX}    % Leave out the initial D/ (it is added
                         % automatically)
%\isbn{XXX-XX-XXXX-XXX-X}


% Set spine width:
\setlength{\adsphdspinewidth}{9mm}

%% Set bleeds
%\setlength{\defaultlbleed}{7mm}
%\setlength{\defaultrbleed}{7mm}

% Set custom cover page
% \setcustomcoverpage{mycoverpage.tex} % mycoverpage.tex is the default

%%% COVER: End settings %%%

% for the nomenclature
\renewcommand{\nomname}{List of Symbols}
\makeatletter
\let\@printnomenclatureorig\@printnomenclature
\def\@printnomenclature[#1]{%
  \cleardoublepage%
  \chaptermark{\nomname}
  \@printnomenclatureorig[#1]
}
\makeatother
\makenomenclature

% for the list of abbreviations.
\newcommand{\glossname}{Abbreviations}
\makeglossary

% To avoid problems, do NOT change the layout of the following two
% commands
\let\printglossaryorig\printglossary
\renewcommand{\printglossary}{%
  \renewcommand{\glossaryname}{\glossname}
  \cleardoublepage%
  \printglossaryorig\chaptermark{\glossname}}


% BibLaTeX
%\usepackage[utf8]{inputenc}
%\usepackage{csquotes}
%\usepackage[
  %hyperref=auto,
  %mincrossrefs=999,
  %backend=biber,
  %style=authoryear-icomp
%]{biblatex}
%\addbibresource{allpapers.bib}

% Fonts
\usepackage{microtype}
\usepackage[T1]{fontenc}
\usepackage{lmodern}  % to make pdf searchable
\usepackage{textcomp} % nice greek alphabet
\usepackage{pifont}   % Dingbats
\usepackage{booktabs}
\usepackage{amssymb,amsthm}
\usepackage{amsmath}


%%%%%%%%%%%%%%%%%%%%%%%%%%%%%%%%%%%%%%%%%%%%%%%%%%%%%%%%%%%%%%%%%%%%%%

\begin{document}

%%%%%%%%%%%%%%%%%%%%%%%%%%%%%%%%%%%%%%%%%%%%%%%%%%%%%%%%%%%%%%%%%%%%%%

\makefrontcoverXII

\maketitle

\frontmatter % to get \pagenumbering{roman}

\includepreface{preface}
\includeabstract{abstract}
\includeabstractnl{abstractnl}

% To create a list of abbreviations, there are 2 options
% 1. manual creation and inclusion of this file
%    \includeabbreviations{abbreviations}
% 2. automatic generation via the glossary package
%    \usepackage{glossary}
%    \makeglossary
%    \glossary{name=MD,description=molecular dynamics}
%    \printglossary
\printglossary

% To create a list of symbols, there are 2 options
% 1. include a manually created nomenclature as a chapter
%    \includenomenclature{nomenclaturechapter}
% 2. automatic generation via the nomencl package
%    \usepackage{nomencl}
%    \makenomenclature
%    \nomenclature[cB]{$c_B(\vec{x})$}{Characteristic function of $B$}
%    \printnomenclature[3cm]
\printnomenclature[1.5cm]

\tableofcontents
\listoffigures
\listoftables

%%%%%%%%%%%%%%%%%%%%%%%%%%%%%%%%%%%%%%%%%%%%%%%%%%%%%%%%%%%%%%%%%%%%%%

\mainmatter % to get \pagenumbering{arabic}

% Show instructions on a separate page
\instructionschapters\cleardoublepage

\includechapter{introduction}
\includechapter{manual} % Remove this chapter

% Insert here your own chapters
% Chapters are expected to be in a tex-file with the given name dot
% tex and in a directory with the given name in the chapters
% directory.

\includechapter{conclusion}

%%%%%%%%%%%%%%%%%%%%%%%%%%%%%%%%%%%%%%%%%%%%%%%%%%%%%%%%%%%%%%%%%%%%%%

\appendix

\includeappendix{myappendix}

%%%%%%%%%%%%%%%%%%%%%%%%%%%%%%%%%%%%%%%%%%%%%%%%%%%%%%%%%%%%%%%%%%%%%%
\backmatter

\includebibliography
\instructionsbibliography
% BibTex
\bibliographystyle{acm}
\bibliography{allpapers}
% BibLatex (comment out lines above and biblatex lines in preamble)
%\printbibliography[title=\bibname]

\includecv{curriculum}

\includepublications{publications}

%\makebackcoverX
\makebackcoverXII

\end{document}

