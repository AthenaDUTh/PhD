%\documentclass[croppedpdf,pagebackref,showinstructions,showtodo]{adsphd}
\documentclass[pagebackref,showinstructions,showtodo]{adsphd}
% Preamble {{{}{

\usepackage[english,dutch]{babel}

\usepackage{nomencl}
\usepackage[style=long,number=none]{glossary} % For list of abbreviations

\title{Multiscale methods for the macroscopic simulation of individual-based models}
\subtitle{Multiscale computing for Dummies}

\author{Yves}{Frederix}

\promotor{Prof.~Dr.~ir.~D.~Roose}
\president{Prof.~Dr.~ir.~Mr.~De~Voorzitter}
\jury{Prof.~Dr.~ir.~Den~Ene\\
      Prof.~Dr.~ir.~Den Andere
}
\externaljurymember{Prof.~Dr.~Het~extern~jurylid}
\externalaffiliation{Ver weg}

\degree{Engineering} % "Doctor in ..."
\faculty{Faculty of Engineering}
\department{Department of Computer Science}
\researchgroup{Scientific Computing Group}
\address{Celestijnenlaan 200A}

\date{september 2010}
\udc{XXX.XX}
\depot{XXXX/XXXX/XX}
\isbn{XXX-XX-XXXX-XXX-X}

% for the nomenclature
\renewcommand{\nomname}{List of Symbols}
\makenomenclature
% for the list of abbreviations. 
\newcommand{\glossname}{Abbreviations}
\makeglossary

% To avoid problems, do NOT change the layout of the following two
% commands
\let\printglossaryorig\printglossary
\renewcommand{\printglossary}{%
  \renewcommand{\glossaryname}{\glossname}
  \printglossaryorig}


% Own commands
% Some useful packages
\usepackage{amssymb,amsthm}
\usepackage{amsmath}

% Some nice colors
\newcommand{\red}[1]{\textcolor{red}{#1}}
\newcommand{\green}[1]{\textcolor{green}{#1}}
\newcommand{\blue}[1]{\textcolor{blue}{#1}}
\newcommand{\cyan}[1]{\textcolor{cyan}{#1}}
\newcommand{\magenta}[1]{\textcolor{magenta}{#1}}
\newcommand{\yellow}[1]{\textcolor{yellow}{#1}}
\newcommand{\orange}[1]{\textcolor{orange}{#1}}
\newcommand{\violet}[1]{\textcolor{violet}{#1}}
\newcommand{\purple}[1]{\textcolor{purple}{#1}}
\newcommand{\brown}[1]{\textcolor{brown}{#1}}
\newcommand{\gray}[1]{\textcolor{gray}{#1}}
\newcommand{\darkgray}[1]{\textcolor{darkgray}{#1}}
\newcommand{\lightgray}[1]{\textcolor{lightgray}{#1}}
\newcommand{\bred}[1]{\textbf{\textcolor{red}{#1}}}
 % defs.tex, autogenerated by Makefile

% }{}}}  <-- Preamble

%%%%%%%%%%%%%%%%%%%%%%%%%%%%%%%%%%%%%%%%%%%%%%%%%%%%%%%%%%%%%%%%%%%%%%

\begin{document}

%%%%%%%%%%%%%%%%%%%%%%%%%%%%%%%%%%%%%%%%%%%%%%%%%%%%%%%%%%%%%%%%%%%%%%

\makefrontcover

\maketitle

\frontmatter % to get \pagenumbering{roman}

\includepreface{preface}
\includeabstract{abstract}

% To create a list of abbreviations, there are 2 options
% 1. manual creation and inclusion of this file
%    \includeabbreviations{abbreviations}
% 2. automatic generation via the glossary package
%    \usepackage{glossary}
%    \makeglossary
%    \glossary{name=MD,description=molecular dynamics}
%    \printglossary
\printglossary

% To create a list of symbols, there are 2 options
% 1. manual creation and inclusion of this file
%    \includenomenclature{nomenclature}
% 2. automatic generation via the nomencl package
%    \usepackage{nomencl}
%    \makenomenclature
%    \nomenclature[cB]{$c_B(\vec{x})$}{Characteristic function of $B$}
%    \printnomenclature[3cm]
\printnomenclature

\tableofcontents
\listoffigures
\listoftables

%%%%%%%%%%%%%%%%%%%%%%%%%%%%%%%%%%%%%%%%%%%%%%%%%%%%%%%%%%%%%%%%%%%%%%

\mainmatter % to get \pagenumbering{arabic}

% Show instructions on a separate page
\instructionschapters\cleardoublepage

\includechapter{introduction}

%%%
% Some dummy code to make sure bibtex does not complain
\red{\cite{FrRo2010Diffusion}}
% Some dummy code to get at least 1 entry in the nomenclature
\red{$\Theta$}
\nomenclature{\red{$\Theta$}}{\red{A nice symbol}}
% Some dummy code to get at least 1 entry in the list of
% abbreviations
\glossary{name=MD,description=molecular dynamics}
%%%

\includechapter{conclusion}

%%%%%%%%%%%%%%%%%%%%%%%%%%%%%%%%%%%%%%%%%%%%%%%%%%%%%%%%%%%%%%%%%%%%%%

\appendix

\includeappendix{myappendix}

%%%%%%%%%%%%%%%%%%%%%%%%%%%%%%%%%%%%%%%%%%%%%%%%%%%%%%%%%%%%%%%%%%%%%%
\backmatter

\includebibliography{allpapers}
\instructionsbibliography

\includecv{curriculum}

\includepublications{publications}

\makebackcover

\end{document}

% vim: tw=70 nocindent expandtab foldmethod=marker foldmarker={{{}{,}{}}}
